\section{Fazit}
\subsection{Zukunft}

Das Thema Sicherheit in Autos nimmt von Jahr zu Jahr zu, bedingt durch die vielen Unfälle, die auf den Straßen passieren.

Doch nicht viele Autos sind mit einer Kamera ausgerüstet.

Die meisten Sicherheitssysteme analysieren nur das Fahrverhalten des Fahrers und zeigen ein 'Kaffee-Symbol' auf dem Display an.

Die Firma Bosch arbeitet zurzeit an einer kompletten Beobachtung des Innenraumes.

So soll bis zum Jahr 2022 ein System auf den Markt gebracht werden, die jede Person im Auto erfassen kann.

Sie erkennt, wenn Kinder auf dem Rücksitz ihren Gurt lösen und warnt den Fahrer oder die Fahrerin.

Sitzt ein Mitfahrer im Fond zu weit nach vorne gelehnt oder gar schräg und mit den Füßen auf dem Nebensitz, können Airbags und Gurtstraffer bei einem Unfall nicht schützen. 

Auch kann mit dem System verhindert werden, dass der Airbag auslöst, wenn eine Babyschale auf dem Vordersitz angebracht wird. 

Ein weiterer wichtiger Punkt für Bosch ist es, zu erfassen ob Kinder oder Tiere unabsichtlich bei sonnigen Wetter im Auto gelassen werden. 

Falls dies erkannt wird, werden sofort die Eltern oder der Rettungsdienst alarmiert.  

In der Zukunft erhofft man sich, mit den neuen Sicherheitssystemen mehr Sicherheit für die Autofahrer. 

So soll Jahr für Jahr versucht werden die Unfallrate zu verringern.\cite{b7}

\subsection{Fazit}

Zusammenfassend, ist das Thema 'Driver Drowsiness Detection System' ein wichtiger Bestandteil unseres Alltages geworden.

Nicht nur weil so verhindert werden kann, dass Personen am Steuer einschlafen,
sondern auch einige weitere Risiken wie Handys oder weiteres, welche den Fahrer ablenken könnten und so eventuell zu einem Unfall führen.

Leider sind noch nicht viele Autos mit so einem System ausgestattet.

So wie im 'Mazda Mx30' beschrieben, dass eine Kamera in Blickrichtung der fahrenden Person schaut. 

Zudem ertönt ein Signal, falls die Person müde wirkt.
 
Dieses ist heutzutage noch kein Standard in den herkömmlichen Autos. 

Es wird in den meisten Autos nur auf die Fahr- und Lenkweise des Fahrers geachtet. 

Dies wird dann nur über ein Kaffee Symbol oder über einer Nachricht auf dem Display angezeigt.

Zwar zeigt es dem Fahrer an, dass er müde sein kann, doch es hindert ihn nicht daran übermüdet weiter zu fahren.

Darum sollten mehr Autos mit dem System vom 'Mazda Mx30' ausgestattet werden, denn das Warnsignal würde den Fahrer konstant signalisieren eine Pause einzulegen.

Mit dem Projekt 'Driver Drowsiness Detection System', wurde gezeigt, dass es für viele Hersteller sehr leicht möglich ist, so ein System in ihren Autos zu implementieren.

Doch leider gibt es noch kein Gesetz, welches vorschreibt eine Kamera im Auto anbringen zu müssen.

Viele Firmen finden die Produktion von solchen Systemen nicht notwendig und arbeiten nicht mit diesem.

Viele Autobesitzer möchten eine solche Kamera nicht in ihrem Auto haben, aus Angst sie könnten überwacht werden oder abgefilmt.

Auch wegen der vielen Gesetze, ist es den Firmen oft nicht möglich Kameras in öffentlichen Räumen zu installieren.

Die Firma Bosch versucht trotz der schwierigen Lage, so ein System im Jahr 2022 zu veröffentlichen.

Zusätzlich zu der Kamera, die auf den Fahrer gerichtet ist, möchten sie auch die anderen Insassen beobachten.

Diese können sich im Auto auch in einer Notlage befinden und der Notruf soll daraufhin gewählt werden oder der Halter des Autos informiert werden.


Schluss folgend ist das Thema 'Driver Drowsiness Detection System', noch nicht ausgreift. 

Da es noch zu viele negative Punkte gibt, die verhindern, dass jeder Autobesitzer so ein System in seinem Auto installieren möchte.\cite{b7}








