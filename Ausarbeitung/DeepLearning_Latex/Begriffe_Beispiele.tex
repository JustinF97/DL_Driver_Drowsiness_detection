\section{Begriffe/Beispiele}

\subsection{Begriffsbestimmung}

Als 'Driver Drowsiness Detection System', im Deutschen Aufmerksamkeits-Assistent, wird eine Müdigkeitserkennung in Kraftfahrzeugen bezeichnet.

Diese Erkennung soll helfen, Unfälle durch schläfrig werdende Fahrer oder Fahrerinnen zu vermeiden. 

Nicht nur das Einschlafen am Steuer ist gefährlich, auch die Müdigkeit reduziert die Fahrtüchtigkeit deutlich. 

Wer schläfrig fährt, schätzt die Geschwindigkeiten falsch ein. 

Man ist unkonzentriert und reagiert ähnlich langsam wie nach dem Konsum von Alkohol.

Im schlimmsten Fall nickt die Fahrerin oder der Fahrer ein. 

Mit dem System soll frühzeitig erkannt werden, ob der Fahrer übermüdet ist und so ein Unfall vermieden werden.\cite{b7}

\subsection{Beispiele}

\subsection{Mazda Mx-30}
Der 'Mazda MX-30' besitzt im Innenraum eine 
Infrarotkamera die Blickrichtung, Augen- und Mundwinkelbewegungen der Person am Steuer überwacht. 

Auch Tageszeit, Geschwindigkeit oder Blinkverhalten werden vom System berücksichtigt und ausgewertet.

Wenn die Person müde wirkt warnt das System die Person über ein akustisches Geräusch.\cite{b2}


\subsection{Andere Autos}
Die meisten Autos verfügen noch nicht über eine eingebaute Innenraum-Kamera.

Sie verwenden in ihrem Sicherheitssystem, nur die Lenkbewegungen und die Position des Fahrzeugs in der Fahrspur. 

Die Software erstellt zu Beginn der Fahrt ein Profil und analysiert das Fahrverhalten ab einer Geschwindigkeit von 65 km/h (zum Beispiel bei VW) oder 80 km/h (bei Mercedes).

Erkennt das System ein Fahrmuster, das auf die Unaufmerksamkeit des Fahrers hindeutet, wird dieser durch ein akustisches Signal oder einem visuellen Hinweis auf dem Zentraldisplay drauf hingewiesen.

Die Idee dahinter, wer am Steuer müde wird, macht öfters kleine Lenkfehler und versucht sie abrupt zu korrigieren. 

Das erkennt die Elektronik, die als Lenkwinkelsensor oftmals als Teil des 'Schleuderschutzes ESP' verbaut ist. 

Auch die Fahrtdauer, das Blinkverhalten oder die Betätigung der Pedale fließen in die Berechnung ein.\cite{b4}
\newline
\newline




 